\documentclass{report}
\usepackage{listings}
\usepackage{xcolor}
\lstset { %
    language=C++,
    backgroundcolor=\color{black!5}, % set backgroundcolor
    basicstyle=\footnotesize,% basic font setting
    breaklines
}
\usepackage{graphicx}
\usepackage{url}
\usepackage{hyperref}
\hypersetup{
    colorlinks=true,
    linkcolor=blue,
    filecolor=magenta,      
    urlcolor=cyan,
}


\title{Development of the Planes Game, a Version of the Battleships Game}
\date{2017-06-09}
\author{Cristian Cucu}
\begin{document}
\maketitle
\newpage
\lstset{language=C++}
\tableofcontents{}

\chapter{Introduction}

\section{Game Description}
The game is a variant of the classical \href{https://en.wikipedia.org/wiki/Battleship_(game)}{battleship game}. The ships will be here called planes and are shown in \ref{fig:board}.
\begin{figure}[h]
  \includegraphics[width = \textwidth]{BoardWithPlanes.png}
  \caption{Board game with 3 planes}
  \label{fig:board}
\end{figure}
The project is organized in two main parts: the game engine and the graphical user interfaces. The game engine is meant to be implemented as a library such as it can be used with a number of graphical frontents. One of the design goals is to make this engine independent from a specific C++ library. Three Graphical User Interfaces (GUI) were programmed up to this point. All of them are based on the Qt framework. They are called PlanesWidget, a simple QWidget approach which also includes debugging tools for the computer's strategy, PlanesGraphicsScene, a GUI based on the QGraphicsScene API, and PlanesQML, based on the QML engine.

\chapter {The Game Engine }
\section{Requirements Analysis}
We need an object that describes a plane object which should at least contain information about the position of the plane on the game grid, the orientation of the plane, the shape of a plane. Additionaly we would need a game board/grid object. It should not be restricted to a specific geometry, should know where each plane is positioned and how many planes there are. Since the game is played against the computer there should be a kind of strategy object that decides the computer's next move. The organization of the game into a series of human vs. computer rounds needs also to be modelled in code.

\section{The Plane Object}
A plane object is defined through the position of its head  (plane front) and its orientation (see \ref{fig:plane_orientations}).  We assume that each plane exists somewhere in its own reference system and in its own rectangular grid - that is a plane object is not explicitly related to a game board object. 

\begin{figure}[h]
  \includegraphics[width = 3cm]{PlaneEastWest.png}
  \includegraphics[width = 3cm]{PlaneWestEast.png}
  \includegraphics[width = 3cm]{PlaneNorthSouth.png}
  \includegraphics[width = 3cm]{PlaneSouthNorth.png}
  \caption{Possible plane orientations}
  \label{fig:plane_orientations}
\end{figure}

\subsection{Class declaration}
\begin{lstlisting}[caption = {Plane Class Declaration},label=plane_declaration]
class Plane
{
public:
    enum Orientation { NorthSouth = 0, SouthNorth = 1, WestEast = 2, EastWest = 3 };

private:
    //plane orientation
    Orientation m_orient;
    //coordinates of the position of the head of the plane
    int m_row, m_col;

public:
    //Various constructors
    Plane();
    Plane(int row, int col, Orientation orient);
    Plane(const PlanesCommonTools::Coordinate2D& qp, Orientation orient);

    //setter and getters
    //gives the planes orientation
    Orientation orientation() const {return m_orient; }
    //gives the plane head's row and column
    int row() const { return m_row; }
    int col() const { return m_col;}
    //sets the plane head position
    void row(int row) { m_row = row; }
    void col(int col) { m_col = col; }
    void orientation(Orientation orient) { m_orient = orient; }
    //gives the coordinates of the plane head
    PlanesCommonTools::Coordinate2D head() const { return PlanesCommonTools::Coordinate2D(m_row, m_col); }

    //operators
    //compares two planes
    bool operator==(const Plane& pl1) const;
    //translates a plane by a 2d translation vector
    Plane operator+(const PlanesCommonTools::Coordinate2D& qp);

    //geometrical transformations
    //clockwise rotation of planes
    void rotate();
    //translation with given offset in a grid with row and col rows and columns
    //if the future head position is not valid do not translate
    void translateWhenHeadPosValid(int offsetX, int offsetY, int row, int col);

    //other utility functions
    //tests whether a point is a plane's head
    bool isHead(const PlanesCommonTools::Coordinate2D& qp) const { return qp == head(); }
    //checks if a certain point on the grid is on the plane
    bool containsPoint(const PlanesCommonTools::Coordinate2D& qp) const;
    //returns whether a plane position is valid (the plane is completely contained inside the grid) in a grid with row and col
    bool isPositionValid(int row, int col) const;
    //generates a random number from 0 and valmax-1
    static int generateRandomNumber(int valmax);
    //displays the plane
    std::string toString() const;
};
\end{lstlisting} 

\subsection {Method Implementation}
\begin{lstlisting}[caption = {Plane Class Methods}, label=plane_implementation]
//Various constructors
Plane::Plane() {
    m_row = 0;
    m_col = 0;
    m_orient = NorthSouth;
}

Plane::Plane(int row, int col, Orientation orient) {
    m_row = row;
    m_col = col;
    m_orient = orient;
}

Plane::Plane(const PlanesCommonTools::Coordinate2D& qp, Orientation orient) {
    m_row = qp.x();
    m_col = qp.y();
    m_orient = orient;
}

//equality operator
bool Plane::operator==(const Plane& pl1) const {
    return ((pl1.m_row == m_row) && (pl1.m_col == m_col) && (pl1.m_orient == m_orient));
}

//Clockwise 90 degrees rotation of the plane
void Plane::rotate() {
    switch(m_orient)
    {
    case NorthSouth:
        m_orient = EastWest;
        break;
    case EastWest:
        m_orient = SouthNorth;
        break;
    case SouthNorth:
        m_orient = WestEast;
        break;
    case WestEast:
        m_orient = NorthSouth;
        break;
    default:
        return;
    }
}

//checks to see if a plane contains a certain point
//uses a PlanePointIterator which enumerates
//all the points on the plane
bool Plane::containsPoint(const PlanesCommonTools::Coordinate2D& qp) const {
    PlanePointIterator ppi(*this);

    while(ppi.hasNext())
    {
        PlanesCommonTools::Coordinate2D qp1 = ppi.next();
        if(qp == qp1)
            return true;
    }

    return false;
}

//Checks to see if the plane is
//in its totality inside a grid
//of size row X col
bool Plane::isPositionValid(int row, int col) const {
    PlanePointIterator ppi(*this);

    while(ppi.hasNext())
    {
        PlanesCommonTools::Coordinate2D qp = ppi.next();
        if(qp.x()<0 || qp.x()>=row)
            return false;
        if(qp.y()<0 || qp.y()>=col)
            return false;
    }

    return true;
}

//utility function
//generates a random number
int Plane::generateRandomNumber(int valmax) {
    double rnd = rand()/ static_cast<double>(RAND_MAX);
    if (rnd==1.0)
        rnd = 0.5;
    int val = static_cast<double>(valmax)*rnd;

    return val;
}

//constructs a string representation of a plane
//used for debugging purposes
std::string Plane::toString() const
{
    std::string toReturn = "";

    toReturn = toReturn + "Plane head: ";
    toReturn = toReturn + std::to_string(m_row);
    toReturn = toReturn + "-";
    toReturn = toReturn + std::to_string(m_col);
    toReturn = toReturn + " oriented: ";

    switch(m_orient)
    {
    case NorthSouth:
        toReturn = toReturn + "NorthSouth";
        break;
    case SouthNorth:
        toReturn = toReturn + "SouthNorth";
        break;
    case EastWest:
        toReturn = toReturn + "EastWest";
        break;
    case WestEast:
        toReturn = toReturn + "WestEast";
        break;
    default:
        ;
    }

    return toReturn;
}

void Plane::translateWhenHeadPosValid(int offsetX, int offsetY, int row, int col) {
    if ((m_row + offsetX < 0) || (m_row + offsetX >= row)) {
        return;
    }

    if ((m_col + offsetY < 0) || (m_col + offsetY >= col)) {
        return;
    }

    m_row += offsetX;
    m_col += offsetY;
}

//implements plane translation
Plane Plane::operator+(const PlanesCommonTools::Coordinate2D& qp) {
    return Plane(this->m_row + qp.x(), this->m_col + qp.y(), this->m_orient);
}
\end{lstlisting}

The implementation of the member functions of the class Plane is trivial. One thing requires an explanation : the class PlanePointIterator. It has to do with the fact that it exists nowhere in the Plane class definition an explicit representation of the points of the game board corresponding to the plane, in other words of the form of a plane. The only things that are defined are the point of origin or the head as well as the orientation of the plane. Thus the class is general allowing the use of any plane form as long as a plane head and orientation are given. In order to allow to work with the plane's form the class PlanePointIterator is used. It encapsulates the form of the plane but also gives a method to \textit iterate on the grid positions corresponding to the plane. 

\subsection {C++ Concepts}

\subsubsection {Class Definition}
Classes are the building blocks of C++ programs. They define properties of program objects as well as operations that can be performed on or with them. A class definition is a program block:
\begin{lstlisting}
class {
.......
};
\end{lstlisting}
Between the two parantheses  are included member variable declarations and method declarations. 

\subsubsection {Member Variable Declaration}
When object is seen as a box with pieces the member variables are the denomination of the placeholders for the pieces. The values of the member variables can be thought as descriptions of the object's state. Each member variable declaration consist of a type name and a variable name. The type name corresponds to the type of the variable, for example simple types as int, char, double or class types.
The listing below shows the member variable declarations in the Plane class.
\begin{lstlisting}
class {
    Orientation m_orient;
    int m_row, m_col;
};
\end{lstlisting}

\subsubsection {Class Method Declaration}
The methods of a class are declared in the corpus of the class and are analogous to function declarations in C: in their simple form they consist of a return type followed by a function name which is followed by function parameters listed in paranthesis after the function name.

\subsubsection {Setters and Getters}
Setters and getters are methods of a class that either change or read a member variable's value. Normally the member variables are not directly visible to the user of an object, but only through the means of class methods of which the simplest are the getters and the setters.
\begin{lstlisting}
    int row() const { return m_row; }
    int col() const { return m_col;}
    void row(int row) { m_row = row; }
    void col(int col) { m_col = col; }
\end{lstlisting}

\subsubsection {Constructor definition}
Constructors of a class are functions that are always called when the object is created. One of their task is the initialization of the member variables.
\begin{lstlisting}
    Plane();
    Plane(int row, int col, Orientation orient);
    Plane(const PlanesCommonTools::Coordinate2D& qp, Orientation orient);
\end{lstlisting}
In the Plane class declaration we declare three constructors. The declaration is similar to a class method declaration except for they have no return types. The three constructors initialize the three member variables of the class Plane with the data that they receive as parameters. 

\subsubsection {Static methods}
Class methods are called with an object of their associated class. Static class methods do not require an object of the associated class. The Plane class has only one such method which generates a random number.
\begin{lstlisting}
static int generateRandomNumber(int valmax);
\end{lstlisting}

\subsubsection {Enums}

\begin{lstlisting}
    enum Orientation {NorthSouth=0, SouthNorth=1, WestEast=2, EastWest=3};
\end{lstlisting}
Enums are basic types for which the variable values are listed at the time of type definition. In our case the new type is called Orientation and variables of this type can have the following values: NorthSouth, SouthNorth, WestEast, EastWest. For enums the associated variable values can be explicitly converted to int. As shown in the example above the conversion to int can be specified directly in the enum definition. Sometimes one desires to avoid such a conversion and to enforce a strict type checking when assigning enum variables to values. In this case one should use the 'enum class' instead of the simple enum.  

\subsubsection {Access Specifiers}
Elements of classes (methods or member variables) can have specified different levels of access to the class user. These are 
\begin{itemize}
\item public, the element is accessible from all functions
\item private, the element is accessible only from the class methods
\item protected, the element is accessible from the class methods or from derived class methods
\end{itemize}  
In the code above the member variables have private access specifiers and such they are not accessible directly from functions other than the class methods. However setter and getter functions are defined with public acces in order to allow access to these members from everywhere in the program. 

\subsubsection {Operators}
Operators such as
\begin{lstlisting}
    bool operator==(const Plane& pl1) const;
    Plane operator+(const PlanesCommonTools::Coordinate2D& qp);
\end{lstlisting}
are syntactic sugar meant to simplify coding. For example, when defining \textit{operator==} one can directly write the comparison \textit{ plane1 == plane2 } with the meaning of a test of equality (if the operator definition respects the semantics of an identity test). 

\subsubsection {What is '*this' ? }
The function containsPoint(..) uses the operator '*this' with the meaning of the object calling the function. More exactly the constructor of the PlanePointIterator object with the name ppi receives as parameter the object on which the containsPoint() function is called.

\begin{lstlisting}
bool Plane::containsPoint(const PlanesCommonTools::Coordinate2D& qp) const {
    PlanePointIterator ppi(*this);

    while(ppi.hasNext())
    {
        PlanesCommonTools::Coordinate2D qp1 = ppi.next();
        if(qp == qp1)
            return true;
    }

    return false;
}
\end{lstlisting}

\subsubsection {Function parameters and their transmission}

An important problem in a C++ program is how the parameters are transmitted (or passed) to functions. There are three important situations: transmission by value, transmission by reference, transmission by const reference.  

When we have a function which has as parameters of simple types (int, double, char, bool) we normally use parameter transmission by value, like in the example below. In passing by value copies of the parameters will be made and the copies will be used by the function. That means that, on one hand, any change that these parameters undergo in function code will not be perceived by the caller of the function, but, on the other hand, a copy operation is involved which for complex data types can be costly.

\begin{lstlisting}
void Plane::translateWhenHeadPosValid(int offsetX, int offsetY, int row, int col) {
    if ((m_row + offsetX < 0) || (m_row + offsetX >= row)) {
        return;
    }

    if ((m_col + offsetY < 0) || (m_col + offsetY >= col)) {
        return;
    }

    m_row += offsetX;
    m_col += offsetY;
}
\end{lstlisting}

For complexer function parameters we want to avoid the copying associated to the passing of the parameters and hence use references types for the parameters. As an example we examine the function containsPoint() from the Plane class:

\begin{lstlisting}
//checks to see if a plane contains a certain point
//uses a PlanePointIterator which enumerates
//all the points on the plane
bool Plane::containsPoint(const PlanesCommonTools::Coordinate2D& qp) const {
    PlanePointIterator ppi(*this);

    while(ppi.hasNext())
    {
        PlanesCommonTools::Coordinate2D qp1 = ppi.next();
        if(qp == qp1)
            return true;
    }

    return false;
}
\end{lstlisting}

The parameter of the function which is of the type PlanesCommonTools::Coordinate2D is passed as a reference, that is a reference to it is given to the function. No copying is involved. Had it not been for the const keyword before the qp parameter the function containsPoint could have modified the value of the parameter in the caller's scope (at the caller). In fact this is a method used to return parameters calculated in functions to the function caller (e.g. when more than one parameters need to be returned by a function). In our concrete case the parameter qp is declared const (see also \ref {Constness}) and the compiler will forbid calling non-const operations on it. Passing parameters as const references is the most employed method of parameter passing as it avoids a copy operation and forbids the changing of the parameter at the caller.  
\section{PlanePointIterator Class}

The PlanePointIterator is responsible with enumerating the points of a plane object and is very often used in other source files. The following section describes the design of this class.

\subsection {Class declaration}

\begin{lstlisting}
//iterates over the points that make a plane
class PlanePointIterator : public PlanesCommonTools::VectorIterator<PlanesCommonTools::Coordinate2D>
{
    Plane m_plane;
public:
    PlanePointIterator(const Plane& pl);

private:
    void generateList();
};
\end{lstlisting}

The definition above says that the PlanePointIterator is defined as a type of PlanesCommonTools::VectorIterator \textless PlanesCommonTools::Coordinate2D \textgreater which is defined (here I anticipate the VectorIterator class definition which is given later) as an iterator over a list of PlanesCommonTools::Coordinate2D objects. It receives a Plane object with which it generates, with the help of the private function generateList, the list of positions occupied by the plane on the game board. It is a type of Java-like iterator, but this will become apparent only after looking at the VectorIterator class. 

\subsection {Method Implementation}
\begin{lstlisting}
PlanePointIterator::PlanePointIterator(const Plane& pl):
    PlanesCommonTools::VectorIterator<PlanesCommonTools::Coordinate2D>(),
    m_plane(pl)
{
    generateList();
}

//the function that generates the list of points
void PlanePointIterator::generateList()
{

    const PlanesCommonTools::Coordinate2D pointsNorthSouth[] = {PlanesCommonTools::Coordinate2D(0, 0), 
    PlanesCommonTools::Coordinate2D(0, 1), PlanesCommonTools::Coordinate2D(-1, 1),
    PlanesCommonTools::Coordinate2D(1, 1), PlanesCommonTools::Coordinate2D(-2, 1), 
    PlanesCommonTools::Coordinate2D(2, 1), PlanesCommonTools::Coordinate2D(0, 2),
    PlanesCommonTools::Coordinate2D(0, 3), PlanesCommonTools::Coordinate2D(-1, 3),
    PlanesCommonTools::Coordinate2D(1, 3)};

    const PlanesCommonTools::Coordinate2D pointsSouthNorth[] = {PlanesCommonTools::Coordinate2D(0, 0), 
    PlanesCommonTools::Coordinate2D(0, -1), PlanesCommonTools::Coordinate2D(-1, -1),
    PlanesCommonTools::Coordinate2D(1, -1), PlanesCommonTools::Coordinate2D(-2, -1),
    PlanesCommonTools::Coordinate2D(2, -1), PlanesCommonTools::Coordinate2D(0, -2), 
    PlanesCommonTools::Coordinate2D(0, -3), PlanesCommonTools::Coordinate2D(-1, -3),
    PlanesCommonTools::Coordinate2D(1, -3)};

    const PlanesCommonTools::Coordinate2D pointsEastWest[] = {PlanesCommonTools::Coordinate2D(0, 0), 
    PlanesCommonTools::Coordinate2D(1, 0), PlanesCommonTools::Coordinate2D(1, -1),
    PlanesCommonTools::Coordinate2D(1, 1), PlanesCommonTools::Coordinate2D(1, -2), 
    PlanesCommonTools::Coordinate2D(1, 2), PlanesCommonTools::Coordinate2D(2, 0), 
    PlanesCommonTools::Coordinate2D(3, 0), PlanesCommonTools::Coordinate2D(3, -1),
    PlanesCommonTools::Coordinate2D(3, 1)};

    const PlanesCommonTools::Coordinate2D pointsWestEast[] = {PlanesCommonTools::Coordinate2D(0, 0), 
    PlanesCommonTools::Coordinate2D(-1, 0), PlanesCommonTools::Coordinate2D(-1, -1),
    PlanesCommonTools::Coordinate2D(-1, 1), PlanesCommonTools::Coordinate2D(-1, -2), 
    PlanesCommonTools::Coordinate2D(-1, 2), PlanesCommonTools::Coordinate2D(-2, 0), 
    PlanesCommonTools::Coordinate2D(-3, 0), PlanesCommonTools::Coordinate2D(-3, 1),
    PlanesCommonTools::Coordinate2D(-3, -1)};

    const int size = 10;
    for(int i = 0; i < size; ++i)
    {
        switch(m_plane.orientation())
        {
            case Plane::NorthSouth:
                m_internalList.push_back(pointsNorthSouth[i] + m_plane.head());
                break;
            case Plane::SouthNorth:
                m_internalList.push_back(pointsSouthNorth[i] + m_plane.head());
                break;
            case Plane::WestEast:
                m_internalList.push_back(pointsWestEast[i] + m_plane.head());
                break;
            case Plane::EastWest:
                m_internalList.push_back(pointsEastWest[i] + m_plane.head());
                break;
            default:
                ;
        }
    }
}
\end{lstlisting}

PlanePointIterator is an iterator that provides access to the positions occupied by the plane on the game board. It's constructor receives as Parameter an object of the class Plane, which defines the position of the plane's head as well as its orientation. The method generateList() contains the positions of cells belonging to a plane relative to the position of its head for each of the four possible plane orientations. To calculate the positions of the cells corresponding to the plane it chooses the right set of relative positions according to the given plane orientation and then it adds them to the absolute position of the plane head.

To create a game with other forms of ships it is sufficient to reimplement the PlanePointIterator class.

\subsection {Implementation of VectorIterator}

The essence of a PlanePointIterator is captured by the VectorIterator class : a PlanePointIterator is a method (defined in the VectorIterator class) to iterate over a list together with a special list, defined with the method generateList().

\begin{lstlisting}
namespace PlanesCommonTools
{
//defines an iterator over a std::vector

template <class T>
class VectorIterator
{
protected:
    std::vector<T> m_internalList;
    int m_idx;

public:
    //constructor
    VectorIterator();

    //sets the position of the iterator before the first  element
    void reset();
    //during an iteration checks to see if there is a next element
    bool hasNext() const;
    //during an iteration returns the next element
    const T& next();
    //returns number of elements
    int itemNo() const;
};

template <class T>
VectorIterator<T>::VectorIterator()
{
    //generates the list of points
    m_internalList.clear();
    //puts the index one before the first element in the list
    reset();
}

template <class T>
void VectorIterator<T>::reset()
{
    m_idx = -1;
}

//during a point iteration checks to see if there is a next point
template <class T>
bool VectorIterator<T>::hasNext() const
{
    return (m_idx < static_cast<int>(m_internalList.size()) - 1);
}

//during an iteration returns the next point
template <class T>
const T& VectorIterator<T>::next()
{
    return  m_internalList[++m_idx];
}

//returns number of points on the plane
template <class T>
int VectorIterator<T>::itemNo() const
{
    return m_internalList.size();
}

\end{lstlisting}

VectorIterator is a parametrizable class (a so called template class) which has only two member variable, a list (implemented through the STL type std::vector) and an index inside the list. 

\subsection {C++ Concepts}
\subsubsection {Iterators}

Iterators are methods to obtain sequential access to the members of a data container (e.g. vector, list, set). No assumption about the order in which the data is retrieved should be made. In the simplest application scenario an iterator is initialized, an action is performed to the data pointed by the iterator, then the iterator is incremented to the next position and so further until the end of data structure is reached. For a C++ style iterator this would look like this:

\begin{lstlisting}
	//initialization of the iterator at the beginning
	auto it = storage.begin();    
	// read data as long as the data end has not been encountered	
	while (it != storage.end()) {  
		//do something with the data pointed to by the iterator		
		do_something(*it);   
		//go to the next position in the data storage 
   		++it;   
	}
\end{lstlisting}

For a Java style iterator this would be:

\begin{lstlisting}
	//initialize iterator with data	
	auto it(storage);     
	//as long as data is available    
	while (it.hasNext()) { 
		//read the data and jump to the next value
		auto d = it.next();  
		//do something with the data
        do_something(d);  
    }
\end{lstlisting}
\subsubsection {Member variable initialization with an initializer list}

\begin{lstlisting}
//constructor
PlanePointIterator::PlanePointIterator(const Plane& pl):
    PlanesCommonTools::VectorIterator<PlanesCommonTools::Coordinate2D>(),
    m_plane(pl)
{
    generateList();
}
\end{lstlisting}

The constructor of the PlanePointIterator class is interesting. It receives as parameter a Plane object. In its body calls the function generateList() to actually create the points corresponding to the plane. Before the function body it makes other two things. First it calls the default constructor of the basis class : PlanesCommonTools::VectorIterator \textless PlanesCommonTools::Coordinate2D \textgreater() and secondly it initializes its member variable m\textunderscore plane with the constructor parameter. This type of member variable initialization is called initialization with initializer list and it is the member variable initialization method recommended for C++. 

\subsubsection {Templates}

Here below is the definition of class VectorIterator, a so called template class. Actually this is a definition of many classes, each for every parameter type T. T is a parameter itself to the class definition. The programmer specifies the type T only when e.g. an object of this class is instantiated. The ability to use template classes is one of the features specific to C++.

Aside of the parameter T the class definition of VectorIterator is a normal class definition. VectorIterator defines a Java-style iterator over a std::vector. Its member variables are a std::vector with type T elements and an index in the std::vector. The class manages these internal variables and offers functionality to return the element corresponding to the index in the list, to jump to the next element in the list, to reset the list index, and to return the number of elements in the list. Observe how the function next() returns an element of type T.

\begin{lstlisting}
template <class T>
class VectorIterator
{
protected:
    std::vector<T> m_internalList;
    int m_idx;

public:
    //constructor
    VectorIterator();

    //sets the position of the iterator before the first  element
    void reset();
    //during an iteration checks to see if there is a next element
    bool hasNext() const;
    //during an iteration returns the next element
    const T& next();
    //returns number of elements
    int itemNo() const;
};

\end{lstlisting}

\subsubsection {References}

Coming back to the function next() in the VectorIterator class definition. It returns a so called 'const reference' to a value of type T. This is advantageous when the objects of type T are big. The reference, specified here with the const T\&, points only to the data which is saved in the std::vector of the VectorIterator. Had we declared the return type as only T, the next function would have created a copy of the next object in the std::vector, which would have existed twice, once in the std::vector inside the VectorIterator and once as the object created when returning from the next() function.

\subsubsection {Namespaces}
\subsubsection {Class derivation}
\subsubsection {Function parameters and their transmission}
\subsubsection {Constness}

Refering again to the function next() in the VectorIterator class definition, we comment on the const keyword. In this particular case it means that the reference returned by the function next() is const, that is the underlying data may not be changed. The compiler verifies that all operations performed on data returned by the function next() are also const, functions that do not change the object on which they are called. Generally the const property can be applied in C++ to functions, function parameters, function return values, member variables. 
\subsubsection {Constants}


\end{document}