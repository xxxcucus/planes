\documentclass{report}
\usepackage{listings}
\usepackage{xcolor}
\lstset { %
    language=C++,
    backgroundcolor=\color{black!5}, % set backgroundcolor
    basicstyle=\footnotesize,% basic font setting
    breaklines
}
\usepackage{graphicx}
\usepackage{url}
\usepackage{hyperref}
\hypersetup{
    colorlinks=true,
    linkcolor=blue,
    filecolor=magenta,      
    urlcolor=cyan,
}
\usepackage[toc,page]{appendix}


\title{Development of the Planes Game, a Version of the Battleships Game}
\date{2022-10-09}
\author{Cristian Cucu}
\begin{document}
\maketitle
\newpage
\lstset{language=C++}
\tableofcontents

\chapter{Introduction}

\section{Game Description}
The game is a variant of the classical \href{https://en.wikipedia.org/wiki/Battleship_(game)}{battleship game}. The ships will be here called planes and are shown in \ref{fig:board}.
\begin{figure}[h]
  \includegraphics[width = \textwidth]{BoardWithPlanes.png}
  \caption{Board game with 3 planes}
  \label{fig:board}
\end{figure}
The project is organized in two main parts: the game engine and the graphical user interfaces. The game engine is meant to be implemented as a library such it can be used with a number of graphical frontents. One of the design goals is to make this engine independent of a specific C++ library. Three Graphical User Interfaces (GUI) based on the Qt framework were programmed up to this point. They are called PlanesWidget, a simple QWidget approach which also includes debugging tools for the computer's strategy, PlanesGraphicsScene, a GUI based on the QGraphicsScene API, and PlanesQML, based on the QML engine. Another graphical user interface was programmed with Java and interfaces through the Java Native Interface to the C++ game engine.

The complete source code of the projects can be found in GitHub (\url{https://github.com/xxxcucus/planes}).

\chapter {The Game Engine }
\section{Requirements Analysis}
We need an object that describes a plane which should at least contain information about the position of the plane on the game grid, the orientation of the plane, the shape of a plane. Additionaly we would need a game board/grid object. It should not be restricted to a specific geometry, should know where each plane is positioned and how many planes there are. Since the game is played against the computer there should be a kind of strategy object that decides the computer's next move. The organization of the game into a series of human vs. computer rounds needs also to be modelled in code.

\include{ThePlaneObject}
\include{ThePlanePointIterator}
\section{The Computer Strategy} \label {Computer_Strategy}

\subsection{Basic Data Structures} \label {Strategy_Data_Structures}

\subsubsection{PlaneOrientationData}

The PlaneOrientationData structure keeps all the information required to justify the position and orientation of a specific plane. It works as follows: the Plane object is stored as a member variable m\_plane, if the position and orientation of the plane are not possible the member variable m\_discarded is set to true, the points on the grid that still need to be searched in order for the plane position and orientation to be completely proven are kept in the member variable m\_pointsNotTested. At the beginning all the points on the plane are in the list m\_pointsNotTested. As the game proceeds different points in m\_pointsNotTested will be tested and depending on what they reveal it can be that m\_discarded is set to true. Anyway after each guess of one point in m\_pointsNotTested the tested point is removed from the list. 

\begin{lstlisting} [caption={PlaneOrientationData Definition}]
struct PlaneOrientationData
{
	//the position of the plane
	Plane m_plane;
	
	//whether this orientation was discarded
	bool m_discarded;
	//points on this plane that were not tested
	//if m_discarded is false it means that all the
	//tested points were hits
	std::vector<PlanesCommonTools::Coordinate2D> m_pointsNotTested;
	
	//default constructor
	PlaneOrientationData();
	//another constructor
	PlaneOrientationData(const Plane& pl, bool isDiscarded);
	//copy constructor
	PlaneOrientationData(const PlaneOrientationData& pod);
	//equals operator
	void operator=(const PlaneOrientationData& pod);
	
	//update the info about this plane with another guess point
	//a guess point is a pair (position, guess result)
	void update(const GuessPoint& gp);
	//verifies if all the points in the current orientation were already checked
	bool areAllPointsChecked();
};

\end{lstlisting}

From the implementation file the following 3 functions are important: \begin{itemize}
	\item the constructor where a PlanePointIterator is used to initialize the m\_pointsNotTested member variable
	\item the update() function which receives a GuessPoint object, which is the result of a guess on the play board. The function updates m\_pointsNotTested and m\_discarded based on this new information.
	\item the function are areAllPointsChecked() which verifies if all points influencing the decision of the plain position and orientation being valid have been tested.
\end{itemize}

\begin{lstlisting} [caption={PlaneOrientationData Implementation}]


//useful constructor
PlaneOrientationData::PlaneOrientationData(const Plane& pl, bool isDiscarded) :
	m_plane(pl),
	m_discarded(isDiscarded)
{
	PlanePointIterator ppi(m_plane);
	
	//all points of the plane besides the head are not tested yet
	ppi.next();

	while (ppi.hasNext())
	{
		m_pointsNotTested.push_back(ppi.next());
	}
}

void PlaneOrientationData::update(const GuessPoint &gp)
{
	//if plane is discarded return
	if (m_discarded)
		return;
	
	//find the guess point in the list of points not tested
	auto it = std::find(m_pointsNotTested.begin(), m_pointsNotTested.end(), PlanesCommonTools::Coordinate2D(gp.m_row, gp.m_col));
	
	//if point not found return
	if (it == m_pointsNotTested.end())
		return;
	
	//if point found
	//if dead and idx = 0 remove the head from the list of untested points
	if (gp.m_type == GuessPoint::Dead && it == m_pointsNotTested.begin())
	{
		m_pointsNotTested.erase(it);
		return;
	}
	
	//if miss or dead discard plane
	if (gp.m_type == GuessPoint::Miss || gp.m_type == GuessPoint::Dead)
		m_discarded = true;
	
	//if hit take point out of the list of points not tested
	if (gp.m_type == GuessPoint::Hit)
		m_pointsNotTested.erase(it);
}

//checks to see that all points on the plane were tested
bool PlaneOrientationData::areAllPointsChecked()
{
	return (m_pointsNotTested.empty());
}

\end{lstlisting}

\subsubsection{HeadData}

In the game of Planes one should guess the position of the plane head and not the exact position and orientation of the corresponding plane. The information about one plane head being at a specific location on the grid is saved in the struct HeadData.

\begin{lstlisting} [caption={HeadData Definition}]
struct HeadData
{
	//size of the grid
	int m_row, m_col;
	//position of the head
	int m_headRow, m_headCol;
	//the correct plane orientation if decided
	int m_correctOrient;
	
	//statistics about the 4 positions with this head
	PlaneOrientationData m_options[4];
	
	HeadData(int row, int col, int headRow, int headCol);
	//update the current data with a guess
	//return true if a plane is confirmed
	bool update(const GuessPoint& gp);
};
\end{lstlisting}

The most important aspect of HeadData is that it contains 4 PlaneOrientationData structures, each for every possible orientation of a plane with a the given plane head position. Additionally the size of the game board is saved, along with the plane head position. If there is enough data to be sure that the plane is in one of the 4 searched positions, m\_correctOrient will be set to the index of the found orientation in the array m\_options. The most important function is the function update() that receives a guess (which is a position on the game board together with a guess result: dead, hit, miss) and updates the knowledge about this plane head's position.

\subsection{Data Available to the Computer}

The computer's decisions are modelled by the class ComputerLogic. It contains informations about possible positions of planes on the game board. Let's look first what happens when new information comes from a guess made by the computer. This is done in the function addData().

The guess is firstly pushed inside two lists m\_guessesList and m\_extendedGuessesList. Then two key data structures are updated with the given guess: the choice map and the head information. Following these two steps we look in the head information to see if we confirmed a plane position, and when we did we add it to m\_guessedPlaneList, eliminate it from m\_headDataList, and update the choice map with the positions of all the plane points on the found plane (with updateChoiceMapPlaneData()).

\begin{lstlisting} [caption={ComputerLogic AddData}]
void ComputerLogic::addData(const GuessPoint& gp)
{
	//add to list of guesses
	m_guessesList.push_back(gp);
	m_extendedGuessesList.push_back(gp);
	
	//updates the info in the array of choices
	updateChoiceMap(gp);
	
	//updates the head data
	updateHeadData(gp);
	
	//checks all head data to see if any plane positions were confirmed
	auto it = m_headDataList.begin();
	
	
	while (it != m_headDataList.end()) {
		//if we decided upon an orientation
		//update the choice map
		//and delete the head data structure
		//append to the list of found planes
		if (it->m_correctOrient != -1)
		{
			Plane pl(it->m_headRow, it->m_headCol, (Plane::Orientation)it->m_correctOrient);
			updateChoiceMapPlaneData(pl);
			m_guessedPlaneList.push_back(pl);
			it = m_headDataList.erase(it);
		} else {
			++it;
		}
	}
}
\end{lstlisting}

The choice map is a vector having one integer element for each board tile and each possible plane orientation. Its values have the following meaning:

\begin{itemize}
	\item when a guess has been made at that position, the choice is -2
	\item when it is impossible to have a plane at that position with the corresponding orientation, the choice is -1
	\item no information is available about this point, means the value 0
	\item a positive value k, means that there are k different sources of information that say that there is a plane at that position and that plane orientation
\end{itemize}

The head information consist in HeadData structures for each of the plane head positions guessed. Its purpose is to identify exactly to which plane orientation corresponds the plane head position.

\subsubsection{Updating the Choice Map}

Updating the choice map works as follows:

\begin{lstlisting} [caption={ComputerLogic UpdateChoiceMap}]
void ComputerLogic::updateChoiceMap(const GuessPoint& gp) {

	//marks all the 4 positions in the choice map as guessed -2
	for(int i = 0;i < 4; i++) {
		Plane plane(gp.m_row, gp.m_col, (Plane::Orientation)i);
		int idx = mapPlaneToIndex(plane);
		m_choices[idx] = -2;
	}
	
	if(gp.m_type == GuessPoint::Dead)
		updateChoiceMapDeadInfo(gp.m_row, gp.m_col);
	
	if(gp.m_type == GuessPoint::Hit)
		updateChoiceMapHitInfo(gp.m_row, gp.m_col);
	
	if(gp.m_type == GuessPoint::Miss)
		updateChoiceMapMissInfo(gp.m_row, gp.m_col);
}
\end{lstlisting}

First the 4 choices corresponding to the guess position are marked with -2, then depending of the result of the guess one of the 3 functions is called: updateChoiceMapDeadInfo(), updateChoiceMapHitInfo(), updateChoiceMapMissInfo(). 

\begin{lstlisting} [caption={ComputerLogic UpdateChoiceMap}]
//updates the choices with info about a dead guess
void ComputerLogic::updateChoiceMapDeadInfo(int row, int col)
{
	//do nothing as everything is done in the updateHeadData function
	//the decision to chose a plane is made in the
	//updateHeadData function
	updateChoiceMapMissInfo(row, col);
}

//updates the choices with info about a hit guess
void ComputerLogic::updateChoiceMapHitInfo(int row,int col)
{
	//for all the plane positions that are valid and that contain the
	//current position increment their score
	
	m_pipi.reset();
	
	while(m_pipi.hasNext()) {
		//obtain index for position that includes Coordinate2D(row,col)
		Plane pl = m_pipi.next();
		PlanesCommonTools::Coordinate2D qp(row, col);
		//add current position to the index to obtain a plane option
		pl = pl + qp;
		
		//if choice is not valid continue to the next position
		if(!pl.isPositionValid(m_row, m_col))
		continue;
		
		//position is valid; check first that it has not
		//being marked as invalid and that increase its score
		
		int idx = mapPlaneToIndex(pl);
		if(m_choices[idx] >= 0)
			m_choices[idx]++;
	}
}

//updates the choices with info about a miss guess
void ComputerLogic::updateChoiceMapMissInfo(int row, int col)
{

	//discard all plane positions that contain this point
	m_pipi.reset();
	
	while(m_pipi.hasNext())
	{
		//obtain index for position that includes Coordinate(row,col)
		Plane pl = m_pipi.next();
		PlanesCommonTools::Coordinate2D qp(row, col);
		//add current position to the index to obtain a plane option
		pl = pl + qp;
		
		//if choice is not valid continue to the next position
		if(!pl.isPositionValid(m_row, m_col))
		continue;
		
		//position is valid; because it includes a miss
		//it must be taken out from the list of choice
		
		int idx = mapPlaneToIndex(pl);
		if(m_choices[idx] >= 0)
			m_choices[idx] = -1;
	}
}
\end{lstlisting}

updateChoiceMapDeadInfo() does the same as updateChoiceMapMissInfo(), leaving the work to updateHeadData(). updateChoiceMapHitInfo() looks at all planes that intersect to the guess location and increments the associated choice in the choice map with one. updateChoiceMapDeadInfo() looks at all the planes that intersect to the guess location and marks the corresponding choice with -1 (impossible position).

\subsubsection {Updating the Head Data}

The head data is data about the position where plane heads were discovered. The structure is created and updated as follows:

\begin{lstlisting} [caption={ComputerLogic UpdateHeadData}]
//updates the head data with a new guess
void ComputerLogic::updateHeadData(const GuessPoint& gp)
{
	//build a list iterator that allows the modification of data
	auto it = m_headDataList.begin();
	
	//updates the head data with the found guess point
	while(it != m_headDataList.end()) {
		it->update(gp);
		++it;
	}
	
	//if the guess point is a head  add a new head data
	//which contains all the knowledge gathered until now
	if(gp.isDead())
	{
		//create a new head data structure
		HeadData hd(m_row, m_col, gp.m_row, gp.m_col);
		
		//update the head data with all the history of guesses
		for(unsigned int i = 0; i < m_extendedGuessesList.size(); i++)
		hd.update(m_extendedGuessesList.at(i));
		
		//append the head data in the list of heads
		m_headDataList.push_back(hd);
	}
}
\end{lstlisting}

On the existing head data, the function update() is called as described in the section \ref{Strategy_Data_Structures} to update the information about the plane position. The purpose is to find the exact orientation of the plane which corresponds to the head position. If the guess made is a dead guess (the head of the plane was guessed), a new head data is created and updated with all the guesses made up to this point of the game.

\subsection {The Computer's Choice}

The computer decides its move based on the information that it has gathered from the previous guesses. 3 possible next moves are generated with the functions makeChoiceFindHeadMode(),  makeChoiceFindPositionMode(),  makeChoiceRandomMode() and of them is statistically chosen as the next move. There is no optimization as to find the optimal next move. makeChoiceFindHeadMode() finds one random point in the choice map which has the highest score and returns that one. makeChoiceFindPositionMode() tries to find the correct plane position corresponding to a guessed plane head. makeChoiceRandomMode() chooses randomly a point on the board which has the score 0 in the choice map. The probability of the makeChoiceRandomMode() is set depending of the difficulty level chosen in the game interface.

\subsection{C++ Concepts}

\subsubsection{Structs}

The types HeadData and PlaneOrientationData are not defined as class but as struct. In C++ a struct is a class where all members variables are by default public. It is in this case convenient to use structs because they allow easy access to the member variables. 
\include{TheGameBoard}
\section{The Game Controller}  \label {Game_Controller}

The game of Planes consists of a series of rounds played by the player against the computer. The player wins when he/she wins more rounds as the computer. This section describes the game controller that coordinates a single round. 

In an initial version of the software the PlaneRound class made use of the signal and slots mechanism of Qt (\ref{Qt_Signals_Slots}) in order to communicate with the graphical user interface. But, in order to simplify the interaction between a graphical user interface written in Java and the PlaneRound, signal and slots were eliminated from PlaneRound and in fact from the whole game engine (more specifically from PlaneGrid). Actually all the dependencies between the game engine and the Qt libraries were eliminated for the above mentioned reason.

\subsection {Class Definition and Important Functions}

The game controller is implemented in the class PlaneRound. Its member variable are as follows:

\begin{lstlisting} [caption={GameController's Member Variable}]
	//whether the computer or the player moves first
	bool m_isComputerFirst;
	//the  game statistics
	GameStatistics m_gameStats;
	
	//the player and computer's grid
	PlaneGrid* m_PlayerGrid;
	PlaneGrid* m_ComputerGrid;
	
	//the list of guesses for computer and player
	std::vector<GuessPoint> m_computerGuessList;
	std::vector<GuessPoint> m_playerGuessList;
	
	//the computer's strategy
	ComputerLogic* m_computerLogic;
\end{lstlisting}

It keeps track of who should make the first move (m\_isComputerFirst), of the score (m\_gameStats) and of the moves made (m\_computerGuessList, m\_playerGuessList). The decision making is left to m\_computerLogic. The two game boards are modelled by m\_PlayerGrid and m\_ComputerGrid.

The API that the PlaneRound offers is (mainly) as follows:

\begin{lstlisting} [caption={PlaneRound's Interface}]

	void initRound();
	void roundEnds();
	
	bool didRoundEnd();
	
	void playerGuess(const GuessPoint& gp, PlayerGuessReaction& pgr);
	void playerGuessIncomplete(int row, int col, GuessPoint::Type& guessRes, PlayerGuessReaction& pgr);
	
	void rotatePlane(int idx);
	void movePlaneLeft(int idx);
	void movePlaneRight(int idx);
	void movePlaneUpwards(int idx);
	void movePlaneDownwards(int idx);
	
	void doneEditing();
	
	int getRowNo() const;
	int getColNo() const;
	
	int getPlaneNo() const;
	int getPlaneSquareType(int i, int j, bool isComputer);
	
	int getPlayerGuessesNo();
	int getComputerGuessesNo();
	
	GuessPoint getPlayerGuess(int idx);
	GuessPoint getComputerGuess(int idx);

\end{lstlisting}

rotatePlane(), movePlaneLeft(), movePlaneRight(), movePlaneUpwards(), movePlaneDownwards(), and doneEditing() are used to position the planes on the player's board in the editing board step of a round. initRound(), roundEnds() and didRoundEnd() initialize, end a round or check if the round has ended. getRowNo(), getColNo(), getPlaneNo() return information about game board size and number of planes used. getPlaneSquareType() gives information about each of the squares of the game board allowing the graphical user interface to draw the square accordingly. getPlayerGuessesNo(), getComputerGuessesNo(), getPlayerGuess(), and getComputerGuess() give information about the number of guesses made in the game, their position and result.

The game controller interacts with the outside (the graphical user interface) through the two functions: playerGuess() and playerGuessIncomplete(). These functions receive the coordinate of a player guess (playerGuessIncomplete()) or the coordinates of a player's guess together with the result of the guess (playerGuess()) and compute the associated computer move. The functions notify if a computer move was generated, if there is a round winner and return the moves statistics.

\begin{lstlisting} [caption={PlaneRound PlayerGuess}]

void PlaneRound::playerGuess(const GuessPoint& gp, PlayerGuessReaction& pgr)
{
	if (m_State != AbstractPlaneRound::GameStages::Game)
	return;
	
	if (m_isComputerFirst) {
		updateGameStatsAndReactionComputer(pgr);
		updateGameStatsAndGuessListPlayer(gp);
	} else {
		updateGameStatsAndGuessListPlayer(gp);
		updateGameStatsAndReactionComputer(pgr);
	}
	
	bool isPlayerWinner = false;
	bool isComputerWinner = false;
	if (roundEnds(isPlayerWinner, isComputerWinner)) {
		if (isPlayerWinner && isComputerWinner) {
			m_gameStats.updateDraws();
			pgr.m_isDraw = true;
		} else {
			m_gameStats.updateWins(isPlayerWinner);
			pgr.m_isDraw = false;
		}
		
		pgr.m_RoundEnds = true;
		m_State = AbstractPlaneRound::GameStages::GameNotStarted;
		pgr.m_isPlayerWinner = isPlayerWinner;
	} else {
		pgr.m_RoundEnds = false;
	}
	
	pgr.m_GameStats = m_gameStats;
}
\end{lstlisting}

Guessing of a computer move works as follows:

\begin{lstlisting} [caption={PlaneRound GuessComputerMove}]

GuessPoint PlaneRound::guessComputerMove()
{
	PlanesCommonTools::Coordinate2D qp;
	//use the computer strategy to get a move
	m_computerLogic->makeChoice(qp);
	
	//use the player grid to see the result of the grid
	GuessPoint::Type tp = m_PlayerGrid->getGuessResult(qp);
	GuessPoint gp(qp.x(), qp.y(), tp);
	
	//add the data to the computer strategy
	m_computerLogic->addData(gp);
	
	//update the computer guess list
	m_computerGuessList.push_back(gp);
	
	return gp;
}

\end{lstlisting}

The function makeChoice() of the ComputerLogic class is used which works as explained in section \ref{Computer_Strategy}. Then the PlayerGrid object is used to evaluate the guess. With the guess position and the guess result the member function addData() from ComputerLogic is called (\ref{Computer_Strategy}). Finally the computer's guess is added to the computer guess list. 

playerGuessIncomplete() is the same as playerGuess() except that it additionally  computes the result of the player's guess at the given coordinates:

\begin {lstlisting} [caption={PlaneRound PlayerGuessIncomplete}]
void PlaneRound::playerGuessIncomplete(int row, int col, GuessPoint::Type& guessRes, PlayerGuessReaction& pgr)
{
	PlanesCommonTools::Coordinate2D qp(col, row);
	guessRes = m_ComputerGrid->getGuessResult(qp);
	GuessPoint gp(qp.x(), qp.y(), guessRes);
	
	playerGuess(gp, pgr);
}
\end{lstlisting}



\chapter {The Graphical User Interface}

\section{Introduction}

We have currently created 4 Graphical User Interfaces (GUIs) three based on C++ and the Qt Framework and one based on Java with the JavaFx library. From the 3 C++ GUIs PlanesWidget uses simple Widget technology of Qt, PlanesGraphicsScene uses the QGraphicsScene/QGraphicsView functionality of Qt, and PlanesQML uses the QML engine of Qt. The chronological order of developing the GUIs was: PlanesWidget, PlanesGraphicsScene, PlanesQML, PlanesJavaFx (not used anymore - marked as obsolete in the repository). As we moved from GUI to GUI we made the separation between the GUI and the GameEngine deeper and deeper. We will comment on these aspects as we present each of the GUI. All of these graphical interfaces are targeted to desktop applications.

Each GUI consist mainly of two parts: a left pane (in most projects denoted as LeftPane) containing the controls for positioning planes in the board editing phase of the game, the moves statistics during a round, the global score and a button allowing to start a new round at the end of a round, and a right pane (in most projects denoted as RightPane) which contains the player's and computer's game boards as well as a help file. In PlanesWidget this structure is not completely respected and we will comment on that as we will present each of the GUI concepts.

An essential aspect of the graphical user interface is how the board games are displayed. 

\include{PlanesWidget}
\section {PlanesGraphicsScene}

\subsection{Main Window}

Similar to the PlanesWidget variant, the main window of the project is an object of a class derived from QMainWindow. In the constructor it creates the View - PlanesGSView - and the controller - PlaneRound - the game engine. 

\begin{figure}[h]
	\includegraphics[width = \textwidth]{PlanesGraphicsScene_BoardEditing_WidgetNames.png}
	\caption{Simplified Layout of PlanesGraphicsScene}
	\label{fig:planesgraphicsscene_boardediting_widgetnames}
\end{figure}

The member variables for the PlanesGSView are:

\begin{lstlisting}

	//PlaneGrid objects manage the logic of a set of planes on a grid
	//as well as various operations: save, remove, search, etc.
	PlaneGrid* m_playerGrid;
	PlaneGrid* m_computerGrid;
	
	//PlaneRound is the object that coordinates the game
	PlaneRound* m_round;
	
	LeftPane* m_LeftPane;
	RightPane* m_RightPane;

\end{lstlisting}

Better as in the PlanesWidget project, the ComputerLogic object is not extracted from the PlaneRound anymore, only the two PlaneGrid objects are, which is still not satisfactory. The widgets composing the GUI are grouped under two principal widgets: LeftPane and RightPane.

The layout of the PlanesGSView is defined as follows:

\begin{lstlisting}
	CustomHorizLayout* hLayout = new CustomHorizLayout(this);
	m_LeftPane = new LeftPane(this);
	m_LeftPane->setMinWidth();
	//m_LeftPane->setSizePolicy(QSizePolicy::Fixed, QSizePolicy::Expanding);
	
	m_RightPane = new RightPane(*m_playerGrid, *m_computerGrid, this);
	m_RightPane->setMinWidth();
	
	hLayout->addWidget(m_LeftPane);
	hLayout->addWidget(m_RightPane);
	setLayout(hLayout);
\end{lstlisting} 

In principle it is a customized horizontal layout (see \ref{custom_qt_layout}) containing the LeftPane and RightPane widgets (see Figure \ref{fig:planesgraphicsscene_boardediting_widgetnames}). 

\subsection{LeftPane}
The LeftPane widget is a QTabWidget, a widget displaying more tabs. There is one tab for each game stage. To each of these tabs corresponds one widget: m\_BoardEditingWidget, m\_GameWidget, m\_StartGameWidget.

The class defines methods to interact when a new game stage starts:

\begin{lstlisting}
    void activateGameTab();
	void activateEditorTab();
	void activateStartGameTab();
\end{lstlisting}

It also defines signals that symbolize clicks on the different buttons in the interface:

\begin{lstlisting}
    void selectPlaneClicked(bool);
	void rotatePlaneClicked(bool);
	void upPlaneClicked(bool);
	void downPlaneClicked(bool);
	void leftPlaneClicked(bool);
	void rightPlaneClicked(bool);
	void doneClicked();
	void startNewGame();
\end{lstlisting}

Finally it defines how the widget reacts to external events, through the following slots:

\begin{lstlisting}

	/**
	* @brief When planes overlap deactivate the done button
	* @param planesOverlap - received info from corresponding signal
	*/
	void activateDoneButton(bool planesOverlap);
	
	/**
	* @brief Activate the game tab when the done button is clicked
	*/
	void doneClickedSlot();
	
	/**
	* @brief activates the editing board tab and the buttons in it
	*/
	void activateEditingBoard();
	
	/**
	* @brief Updates the statistics in the left pane
	*/
	void updateGameStatistics(const GameStatistics& gs);
	
	/**
	* @brief
	* Hide the other tabs.
	*/
	void endRound(bool isPlayerWinner);

\end{lstlisting}

\subsection{RightPane}

The RightPane is a QTabWidget as well with 3 tabs: player's game board, computer's game board and the help page.

The RightPane emits two signals:

\begin{lstlisting}
	void planePositionNotValid(bool);
	void guessMade(const GuessPoint& gp);
\end{lstlisting}

and reacts to signals with the slots:

\begin{lstlisting}
    void resetGameBoard();
	void selectPlaneClicked(bool);
	void rotatePlaneClicked(bool);
	void upPlaneClicked(bool);
	void downPlaneClicked(bool);
	void leftPlaneClicked(bool);
	void rightPlaneClicked(bool);

	/**
	* @brief Switch to computer tab and start looking for planes.
	* Change the internal state of the player's and computer's boards to game stage
	*/
	void doneClicked();
	
	/**
	* @brief Display winner message in the player and computer boards.
	* Block mouse click events in the computer board.
	*/
	void endRound(bool isPlayerWinner);
	void startNewGame();
	void showComputerMove(const GuessPoint& gp);
\end{lstlisting}

The core of the RightPane class are the two game boards, implemented in the classes: PlayerBoard and ComputerBoard.

\subsection{The Game Boards}

Both PlayerBoard and ComputerBoard derive from the class GenericBoard, that provides the tools to interact with a PlaneGrid object from the game controller in order to display the planes and guesses throughout the game.

PlayerBoard and ComputerBoard in PlanesGraphicsScene correspond to GameRenderArea from the PlanesWidget. GenericBoard in PlanesGraphicsScene corresponds to BaseRenderArea in PlanesWidget.

The most important improvement brought by GenericBoard when compared to BaseRenderArea is that it is not a QWidget where the paintEvent() method was overriden, but it uses a specialized framework of Qt: the GraphicsScene/GraphicsView framework.

The display of the board works as follows: the GenericBoard class keeps a collection of GridSquare objects, one for each square of the game board. The GridSquare objects can be parts of the plane, empty squares, plane head, and guesses(hit, miss or dead). When the program is started GridSquare objects are created for every square on the player and computer game boards. Throughout the game, the GameBoard object reads the state of the game from the game engine (PlaneRound), and updates the GridSquare objects accordingly.

\subsection{Qt Concepts}

\subsubsection{Implementing a Custom Layout} \label{custom_qt_layout}

A custom layout was used for the PlanesGSView class:

\begin{lstlisting}
int CustomHorizLayout::count() const
{
	return m_ItemsList.size();
}

QLayoutItem* CustomHorizLayout::itemAt(int idx) const
{
	// QList::value() performs index checking, and returns 0 if we are
	// outside the valid range
	return m_ItemsList.value(idx);
}

QLayoutItem* CustomHorizLayout::takeAt(int idx)
{
	// QList::take does not do index checking
	return idx >= 0 && idx < m_ItemsList.size() ? m_ItemsList.takeAt(idx) : 0;
}

void CustomHorizLayout::addItem(QLayoutItem* item)
{
	if (count() < 2)
		m_ItemsList.append(item);
}

CustomHorizLayout::~CustomHorizLayout()
{
	QLayoutItem *item;
	while ((item = takeAt(0)))
	delete item;
}

void CustomHorizLayout::setGeometry(const QRect &r)
{
	QLayout::setGeometry(r);
	
	///only two widgets can lie in this layout
	if (count() != 2)
	return;
	
	QList<int> scalHoriz;
	
	scalHoriz.append(20);
	scalHoriz.append(80);
	
	int w = r.width() - (count() + 1) * spacing();
	int i = 0;
	int curX = 0;
	while (i < count()) {
		QLayoutItem *o = m_ItemsList.at(i);
		int wtemp = (w * scalHoriz[i]) / 100;
		int htemp = r.height();
		if (i == 0) {
			wtemp = std::max(wtemp, o->minimumSize().width());
			htemp = std::max(r.height() / 2, o->minimumSize().height());
		} else {
			wtemp = std::max(r.width() - curX - spacing(), o->minimumSize().width());
		}
		QRect geom(r.x() + curX + (i + 1) * spacing(), r.y(), wtemp, htemp);
		o->setGeometry(geom);
		++i;
		curX += wtemp;
	}
}

QSize CustomHorizLayout::sizeHint() const
{
	QSize s(0,0);
	int n = m_ItemsList.count();
	if (n > 0)
		s = QSize(1000, 600); //start with a nice default size
	int i = 0;
	while (i < n) {
		QLayoutItem *o = m_ItemsList.at(i);
		s = s.expandedTo(o->sizeHint());
		++i;
	}
	return s + n*QSize(spacing(), spacing());
}

QSize CustomHorizLayout::minimumSize() const
{
	QSize s(0,0);
	int n = m_ItemsList.count();
	int i = 0;
	while (i < n) {
		QLayoutItem *o = m_ItemsList.at(i);
		s = s.expandedTo(o->minimumSize());
		++i;
	}
	return s + n*QSize(spacing(), spacing());
}

\end{lstlisting}

The class CustomHorizLayout extends QLayout. It is a layout that accepts only two widgets, the condition is defined in the addItem() function. The most important function is setGeometry(). Here we define that given a total layout size w, the width of the LeftPane widget (the first to be added in the layout) is

\begin{lstlisting}
	std::max((w * 20) / 100;, o->minimumSize().width());
\end{lstlisting}

where o is the LeftPane widget, which has defined a minimumSize() based on the display size of strings that are to be displayed in the widget.

The RightPane should be displayed directly near the LeftPane with a width of 

\begin{lstlisting}
	std::max(r.width() - curX - spacing(), o->minimumSize().width());
\end{lstlisting}

That means it will take the remaining size width as long as this width is not smaller as the minimum width of the RightPane widget (which is defined in the minimumSize() function for the widget)

The other functions of the class are standard functions, like they are shown in tutorials on the Qt website.

\subsubsection {GraphicsScene - GraphicsView Framework of Qt}

The framework allows to create a space (or scene) (QGraphicsScene) where objects (QGraphicsItems) are displayed. The objects can be added to the scene with a addItem() function. During their lifetime objects can move on the scene, be deleted or change their appearance.

In PlanesGraphicsScene the graphic items objects are of the class GridSquare, which derives from QGraphicsItem, and in the derived class PlayAreaGridSquare. In order to define the appearance of the graphic objects one has to override the function paint() from the QGraphicsItem class.

For example in the PlayAreaGridSquare:

\begin{lstlisting}
void PlayAreaGridSquare::paint(QPainter* painter, const QStyleOptionGraphicsItem* option, QWidget* widget) {
	Q_UNUSED(option)
	Q_UNUSED(widget)
	
	painter->setRenderHint(QPainter::Antialiasing);
	drawCommonGraphics(painter);
	
	if (m_ShowGuesses) {
		switch(m_Status) {
		case GameStatus::Empty:
		break;
		case GameStatus::PlaneGuessed:
		drawPlaneGuessed(painter);
		break;
		case GameStatus::PlaneHeadGuessed:
		drawPlaneHeadGuessed(painter);
		break;
		case GameStatus::TestedNotPlane:
		drawTestedNotPlane(painter);
		break;
		}
	}
}

void PlayAreaGridSquare::drawCommonGraphics(QPainter* painter)
{
	if (m_Selected) {
		painter->fillRect(boundingRect(), Qt::blue);
	} else {
		if (m_ShowPlane) {
			switch(m_Type) {
			case Type::Empty:
			painter->fillRect(boundingRect(), Qt::white);
			break;
			case Type::PlaneHead:
			painter->fillRect(boundingRect(), Qt::green);
			break;
			case Type::Plane:
			painter->fillRect(boundingRect(), m_Color);
			break;
			}
		} else {
			painter->fillRect(boundingRect(), Qt::white);
		}
	}
	
	//    if ((m_ShowPlane && m_Type != Type::Empty) || m_Selected)
	//        painter->setPen(Qt::magenta);
	//    else
	painter->setPen(Qt::black);
	painter->drawRect(boundingRect());
}

void PlayAreaGridSquare::drawPlaneGuessed(QPainter* painter)
{
	QPainterPath fillPath;
	fillPath.moveTo(0, m_Width / 2);
	fillPath.lineTo(m_Width / 2, 0);
	fillPath.lineTo(m_Width, m_Width / 2);
	fillPath.lineTo(m_Width / 2, m_Width);
	fillPath.lineTo(0, m_Width / 2);
	painter->fillPath(fillPath, Qt::darkRed);
}

void PlayAreaGridSquare::drawPlaneHeadGuessed(QPainter* painter)
{
	painter->setPen(Qt::red);
	painter->drawLine(0, 0, m_Width, m_Width);
	painter->drawLine(0, m_Width, m_Width, 0);
}

void PlayAreaGridSquare::drawTestedNotPlane(QPainter* painter)
{
	QPainterPath fillPath;
	fillPath.addEllipse(m_Width / 4, m_Width / 4, m_Width / 2, m_Width / 2);
	painter->fillPath(fillPath, Qt::red);
}

\end{lstlisting}
\include{PlanesQML}

\begin{appendices}
	\section{Configure CMake}

After you have cloned the repository from GitHub, you need to install Qt.

In order to prepare the building process you have to configure CMake either in command prompt or from a GUI CMake program. In Windows as well as in Linux CMake-gui is a GUI CMake program. In \ref{fig:cmakegui_planes} you can see a screenshot of the cmake-gui program window after the path to the project (the folder where the main CMakeLists.txt lies) and the folder where the project should be built are set.


\begin{figure}[h]
	\includegraphics[width = \textwidth]{PlanesCPlusPlus_CMAKE_GUI_Window.png}
	\caption{CMake-GUI Window for project Planes}
	\label{fig:cmakegui_planes}
\end{figure}

You have here several options which need to be set:

\begin{itemize}
	\item CMAKE\_INSTALL\_PREFIX defines the path where the project binaries will be built
	\item Qt*\_DIR are the paths where the various Qt libraries are to be found by the project. Usually you have to define Qt6\_DIR and then the others will be setup automatically.
	\item With\_Asan will activate Address Sanitizer support
	\item With\_Java has to be active when building for the Android application (this option is no longer used).
\end{itemize}

CMake GUI uses so called generators to define which toolchain is used for compilation. These need to be set for example for Visual Studio (as seen in \ref{fig:cmakegui_planes}), gcc, MinGW or any other toolchain you use to build the project.

To perform project configuration click "Configure" and then "Generate".

\section {Binaries}

\subsection{Windows - Visual Studio}

For the Visual Studio generator, after "Generate" is clicked, a Visual Studio solution file is created (.sln). You can use this solution file to open the project. You need to build the INSTALL target inside the project to generate the binaries and move them along with the required dependencies in the installation path. 

\subsection {Windows - MinGW} 

You have to choose the MINGW generator before defining the CMAKE variables in the CMAKE-Gui. You have to run Configure and Generate as well as with the Visual Studio generator. After you have done that you have to go to Windows Command Prompt in the build path. There you should give the following command: cmake.exe --build . This will use MinGW's make program to build the program. To install the binaries along with their dependencies in the installation path (configured in CMAKE-Gui with CMAKE\_INSTALL\_PREFIX) run cmake.exe --build install .

The Windows binaries of Planes are generated using the MinGW generators. Besides the Qt Libraries the installation needs the 3 dlls from the MinGW distribution. These are

\begin{itemize}
	\item libgcc\_s\_seh: Exception handling
	\item libstdc++: C++ Standard Library (C/C++ library functions etc.)
	\item libwinpthread: PThreads implementation on Windows (Threading)
\end{itemize}

In order for these files to be found one needs to configure the path to the MinGW Installation. Normally one can find it in the Qt Installation.
This is done with an environment variable: MINGW\_HOME.

PlanesGraphicsScene makes use in the Multiplayer module of SSL.The two dlls:

\begin{itemize}
	\item libcrypto-1\_1-x64.dll
	\item libssl-1\_1-x64.dll
\end{itemize}

are needed for the project to be able to connect to the game server.

The two dlls are taken from the OpenSSL Installation folder. For that one must specify the path to this folder in the environment variable: OPENSSL\_HOME.

\subsection {Linux - GCC}

Configuration of the project with CMake-Gui is the same. I am usually doing only the setting of the CMAKE variables with the gui programs and then stop CMake-Gui. This is because CMake-Gui for Linux has not been updated in a while and I want to use always the latest CMake version. 

After closing CMake-GUI go to the build folder in terminal. Run cmake .. then make and make install to install the binaries and their dependencies in the installation path.

For the PlanesGraphicsScene one should download and compile locally OpenSSL and set an environment variable OPENSSL\_HOME.

The libraries that are used from the OpenSSL folder are:

\begin{itemize}
	\item libcrypto.so.1.1
	\item libssl.so.1.1
\end{itemize}
\end{appendices}


\end{document}