\section{The Game Controller}  \label {Game_Controller}

The game of Planes consists of a series of rounds played by the player against the computer. The player wins when he/she wins more rounds as the computer. This section describes the game controller that coordinates a single round. 

In an initial version of the software the PlaneRound class made use of the signal and slots mechanism of Qt (\ref{Qt_Signals_Slots}) in order to communicate with the graphical user interface. But, in order to simplify the interaction between a graphical user interface written in Java and the PlaneRound, signal and slots were eliminated from PlaneRound and in fact from the whole game engine (more specifically from PlaneGrid). Actually all the dependencies between the game engine and the Qt libraries were eliminated for the above mentioned reason.

\subsection {Class Definition and Important Functions}

The game controller is implemented in the class PlaneRound. Its member variable are as follows:

\begin{lstlisting} [caption={GameController's Member Variable}]
	//whether the computer or the player moves first
	bool m_isComputerFirst;
	//the  game statistics
	GameStatistics m_gameStats;
	
	//the player and computer's grid
	PlaneGrid* m_PlayerGrid;
	PlaneGrid* m_ComputerGrid;
	
	//the list of guesses for computer and player
	std::vector<GuessPoint> m_computerGuessList;
	std::vector<GuessPoint> m_playerGuessList;
	
	//the computer's strategy
	ComputerLogic* m_computerLogic;
\end{lstlisting}

It keeps track of who should make the first move (m\_isComputerFirst), of the score (m\_gameStats) and of the moves made (m\_computerGuessList, m\_playerGuessList). The decision making is left to m\_computerLogic. The two game boards are modelled by m\_PlayerGrid and m\_ComputerGrid.

The API that the PlaneRound offers is (mainly) as follows:

\begin{lstlisting} [caption={PlaneRound's Interface}]

	void initRound();
	void roundEnds();
	
	bool didRoundEnd();
	
	void playerGuess(const GuessPoint& gp, PlayerGuessReaction& pgr);
	void playerGuessIncomplete(int row, int col, GuessPoint::Type& guessRes, PlayerGuessReaction& pgr);
	
	void rotatePlane(int idx);
	void movePlaneLeft(int idx);
	void movePlaneRight(int idx);
	void movePlaneUpwards(int idx);
	void movePlaneDownwards(int idx);
	
	void doneEditing();
	
	int getRowNo() const;
	int getColNo() const;
	
	int getPlaneNo() const;
	int getPlaneSquareType(int i, int j, bool isComputer);
	
	int getPlayerGuessesNo();
	int getComputerGuessesNo();
	
	GuessPoint getPlayerGuess(int idx);
	GuessPoint getComputerGuess(int idx);

\end{lstlisting}

rotatePlane(), movePlaneLeft(), movePlaneRight(), movePlaneUpwards(), movePlaneDownwards(), and doneEditing() are used to position the planes on the player's board in the editing board step of a round. initRound(), roundEnds() and didRoundEnd() initialize, end a round or check if the round has ended. getRowNo(), getColNo(), getPlaneNo() return information about game board size and number of planes used. getPlaneSquareType() gives information about each of the squares of the game board allowing the graphical user interface to draw the square accordingly. getPlayerGuessesNo(), getComputerGuessesNo(), getPlayerGuess(), and getComputerGuess() give information about the number of guesses made in the game, their position and result.

The game controller interacts with the outside (the graphical user interface) through the two functions: playerGuess() and playerGuessIncomplete(). These functions receive the coordinate of a player guess (playerGuessIncomplete()) or the coordinates of a player's guess together with the result of the guess (playerGuess()) and compute the associated computer move. The functions notify if a computer move was generated, if there is a round winner and return the moves statistics.

\begin{lstlisting} [caption={PlaneRound PlayerGuess}]

void PlaneRound::playerGuess(const GuessPoint& gp, PlayerGuessReaction& pgr)
{
	if (m_State != AbstractPlaneRound::GameStages::Game)
	return;
	
	if (m_isComputerFirst) {
		updateGameStatsAndReactionComputer(pgr);
		updateGameStatsAndGuessListPlayer(gp);
	} else {
		updateGameStatsAndGuessListPlayer(gp);
		updateGameStatsAndReactionComputer(pgr);
	}
	
	bool isPlayerWinner = false;
	bool isComputerWinner = false;
	if (roundEnds(isPlayerWinner, isComputerWinner)) {
		if (isPlayerWinner && isComputerWinner) {
			m_gameStats.updateDraws();
			pgr.m_isDraw = true;
		} else {
			m_gameStats.updateWins(isPlayerWinner);
			pgr.m_isDraw = false;
		}
		
		pgr.m_RoundEnds = true;
		m_State = AbstractPlaneRound::GameStages::GameNotStarted;
		pgr.m_isPlayerWinner = isPlayerWinner;
	} else {
		pgr.m_RoundEnds = false;
	}
	
	pgr.m_GameStats = m_gameStats;
}
\end{lstlisting}

Guessing of a computer move works as follows:

\begin{lstlisting} [caption={PlaneRound GuessComputerMove}]

GuessPoint PlaneRound::guessComputerMove()
{
	PlanesCommonTools::Coordinate2D qp;
	//use the computer strategy to get a move
	m_computerLogic->makeChoice(qp);
	
	//use the player grid to see the result of the grid
	GuessPoint::Type tp = m_PlayerGrid->getGuessResult(qp);
	GuessPoint gp(qp.x(), qp.y(), tp);
	
	//add the data to the computer strategy
	m_computerLogic->addData(gp);
	
	//update the computer guess list
	m_computerGuessList.push_back(gp);
	
	return gp;
}

\end{lstlisting}

The function makeChoice() of the ComputerLogic class is used which works as explained in section \ref{Computer_Strategy}. Then the PlayerGrid object is used to evaluate the guess. With the guess position and the guess result the member function addData() from ComputerLogic is called (\ref{Computer_Strategy}). Finally the computer's guess is added to the computer guess list. 

playerGuessIncomplete() is the same as playerGuess() except that it additionally  computes the result of the player's guess at the given coordinates:

\begin {lstlisting} [caption={PlaneRound PlayerGuessIncomplete}]
void PlaneRound::playerGuessIncomplete(int row, int col, GuessPoint::Type& guessRes, PlayerGuessReaction& pgr)
{
	PlanesCommonTools::Coordinate2D qp(col, row);
	guessRes = m_ComputerGrid->getGuessResult(qp);
	GuessPoint gp(qp.x(), qp.y(), guessRes);
	
	playerGuess(gp, pgr);
}
\end{lstlisting}

